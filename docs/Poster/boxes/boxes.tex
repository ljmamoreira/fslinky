\documentclass{article}
\usepackage[utf8]{inputenc}
\usepackage[portuguese]{babel}
\usepackage[T1]{fontenc}
\usepackage[a4paper,margin=]{geometry}
\usepackage{icomma,amsmath}
\begin{document}
\section*{Motivação}
Vários vídeos disponíveis na plataforma YouTube mostram a queda de uma mola
elástica a partir de uma situação de repouso estático em que ela se encontra na
vertical, suspensa de uma das suas extremidades. Estes vídeos são interessantes
poque mostram a extremidade inferior da mola como que a aguardar que a
extremidade superior a atinja, antes de começar o seu movimento de queda
propriamente dito.

A explicação deste comportamento é dada pela teoria da elasticidade. A onda de
deformação gerada na extremidade superior da mola quando é solta propaga-se
longitudinalmente com uma velocidade finita, e só quando atinge a extremidade
inferior, alterando aí o estado de deformação inicial, se altera o equilíbrio de
forças (peso e força elástica) que mantinham esta extremidade em repouso. 
\end{document}
