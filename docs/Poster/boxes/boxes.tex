\documentclass{article}
\usepackage[utf8]{inputenc}
\usepackage[portuguese]{babel}
\usepackage[T1]{fontenc}
\usepackage[a4paper,margin=]{geometry}
\usepackage{icomma,amsmath,tikz}
\usetikzlibrary{decorations.pathmorphing,patterns}
\begin{document}
\section*{Motivação / Introdução}
  Vários vídeos disponíveis na plataforma YouTube mostram a queda de uma mola
  elástica a partir de uma situação de repouso estático em que ela se encontra
  na vertical, suspensa de uma das suas extremidades. Estes vídeos são
  interessantes poque mostram a extremidade inferior da mola como que a aguardar
  que a extremidade superior a atinja, antes de começar o seu movimento de queda
  propriamente dito. 
	
  A explicação deste comportamento é dada pela teoria da elasticidade de um meio
  contínuo. A onda de deformação gerada na extremidade superior da mola quando é
  solta propaga-se longitudinalmente com uma velocidade finita, e só quando
  atinge a extremidade inferior, alterando aí o estado de deformação inicial, se
  modifica o equilíbrio de forças (peso e força elástica) que mantinham esta
  extremidade em repouso.
	
  Claramente, o modelo elementar de mola ideal, em que se despreza a sua massa,
  é insuficiente para enquadrar esta explicação, uma vez que não tendo massa,
  (1)~a mola não fica sujeita à gravidade, ou seja, não cai e (2)~a sua
  deformação é sempre uniforme, pelo que a força sobre a extremidade inferior
  altera-se instantaneamente assim que a extremidade superior inicia a sua
  queda. Mas será possível dar conta deste comportamento das molas reais
  considerando molas ideais com massas distribuídas regularmemente ao longo do
  seu comprimento? Com este trabalho pretendeu-se dar resposta a esta questão.

\section*{Formalismo}
\begin{tikzpicture}[baseline=(current bounding box.north),scale=0.9]
  \small
  % Supporting structure
  \fill [pattern = north west lines] (-0.75,0) rectangle ++(1.5,.2);
  \draw[thick] (-0.75,0) -- ++(1.5,0);
  \draw (0.80,-4.7) -- (0.90,-4.7);
  \draw [->] (0.85,-4.7) -- +(0,1) node[above]{$y$};
  % Spring
  \tikzset
  {
    spring/.style n args={2}{decorate,
                   decoration={coil, aspect=0.4,
                               segment length=#1,
                               amplitude = 2mm,
                               %pre length=2mm,
                               %post length=#2
                              }
                            }
  }
  \draw (0,0) -- (0,-0.4) coordinate(a);
  \draw [spring={1.5mm}{0.5mm}] ([yshift=-1.8mm]a) -- ++(0,-1.45)
    coordinate[yshift=-1.65mm](b);
  \draw [spring={1.0mm}{0mm}] ([yshift=-1.8mm]b)-- ++(0,-0.61) coordinate (c);
  \path (a) node[circle, draw, fill=white,inner sep=1]{\rule{0mm}{3mm}}
    node[left, inner sep=8]{0};
  \path (b) node[circle, draw, fill=white,inner sep=1]{\rule{0mm}{3mm}}
    node[left, inner sep=8]{1};
  %\fill (c) circle(0.1);
  \path (c) node[inner sep=0,below]{$\vdots$};
  \draw [spring={0.7mm}{1mm}] (0,-3.75) -- ++(0,-0.405)
    node[below,circle,fill=white, draw, inner sep=0]{\rule{0mm}{3mm}}
    node[shift={(-7mm,-2mm)}]{$N-1$};
\end{tikzpicture}\hfill
\begin{minipage}[t]{0.8\linewidth}
Mola ideal com comprimento natural $L$ com constante elástica $K$, com $N$
partículas pontuais iguais com massa $m=M/N$, dispostas regularmente ao longo da
mola.
\begin{align*}
  l&=L/(N-1)&&\rightarrow\text{Comprimento de cada segmento}\\
  k&=K(N-1)&& \rightarrow\text{Constante elástica de cada segmento}\\
  m&=M/N    &&\rightarrow\text{Massa de cada partícula}
\end{align*}
Posições de equilíbrio iniciais:
\begin{equation*}
  y_i^0=y_0^0-i\left[l+\left(N-\frac{i+1}{2}\right)\frac{mg}{k}\right],\quad i>0
\end{equation*}
\end{minipage}
Acelerações ($x_i=y_i-y_i^0$):
\begin{equation}
  \begin{aligned}
    \ddot x_0 &=-Ng+\omega^2(x_1-x_0)\\
    \ddot x_i &= \omega^2(x_{i-1}-2x_i+x_{i+1}),\qquad i=1, \ldots, N-2\\
    \ddot x_{N-1} &=\omega^2(x_{N-2}-x_{N-1})
  \end{aligned}
\end{equation}
Este sistema de equações foi resolvido em Python/Numpy, usando a função
\texttt{solve\_ivp} da biblioteca SciPy.

\section*{Procedimento experimental}
Para verificar a validade das soluções obtidas numericamente, estudou-se o
movimento de queda de uma mola de aço (slinky) com bolas de ténis dispostas
regularmente ao longo do seu comprimento. Esse movimento foi filmado a 120 fps e
a posição das bolas foi obtida fotograma a fotograma usando o programa Tracker
\cite{Tracker}. Verificou-se um bom acordo entre os resultados númericos e os
valores observados deste modo 

\section*{Resultados}
Na Figura A estão representados gráficos da posição calculada e observada de
cada massas como funções do tempo, para o caaso $N=3$. É patente o bom acordo
entre os valores calculados e observados. As discrepâncias mais significativas
ocorrem no final do intervalo de tempo representado, quando as espiras da mola
se comprimem umas contra as outras, devido ao movimento de queda mais rápido das
massas mais acima, efeito que altera a dinâmica puramente elástica e gravítica
do modelo numérico.

Na Figura B está representada a posição da massa na extremidade inferior da
mola, como função do tempo, para diferentes números (2, 3 e 4) de massas
dispostas na mola. Nota-se quanto maior for o número de massas, maior é o
atraso no início do movimento da última massa. Este facto sugere que o
comportamento de queda das molas demonstrado nos vídeos referidos resulta da
densidade de massa finita das molas reais.

\section*{Conclusões}

\end{document}
