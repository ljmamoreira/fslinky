\documentclass{article}
\usepackage[utf8]{inputenc}
\usepackage[portuguese]{babel}
\usepackage[T1]{fontenc}
\usepackage[a4paper,margin=]{geometry}
\usepackage{icomma,amsmath}
\begin{document}
\section*{Motivação / Introdução}
Vários vídeos disponíveis na plataforma YouTube mostram a queda de uma mola
elástica a partir de uma situação de repouso estático em que ela se encontra na
vertical, suspensa de uma das suas extremidades. Estes vídeos são interessantes
poque mostram a extremidade inferior da mola como que a aguardar que a
extremidade superior a atinja, antes de começar o seu movimento de queda
propriamente dito. 

A explicação deste comportamento é dada pela teoria da elasticidade de um meio
contínuo. A onda de deformação gerada na extremidade superior da mola quando é
solta propaga-se longitudinalmente com uma velocidade finita, e só quando atinge
a extremidade inferior, alterando aí o estado de deformação inicial, se altera o
equilíbrio de forças (peso e força elástica) que mantinham esta extremidade em
repouso.

Claramente, o modelo elementar de mola ideal, em que se despreza a sua massa, é
insuficiente para enquadrar esta explicação, uma vez que não tendo massa, (1)~a
mola não fica sujeita à gravidade, ou seja, não cai e (2)~a sua deformação é
sempre uniforme, pelo que a força sobre a extremidade inferior altera-se
instantaneamente assim que a extremidade superior inicia a sua queda. 

Com este trabalho, tentou-se descrever o movimento de queda das molas reais
usando o modelo de mola ideal, com massas iguais distribuídas uniformemente
sobre o seu comprimento.

\section*{Formalismo}

\section*{Procedimento experimental}

\section*{Resultados}

\section*{Conclusões}

\end{document}
