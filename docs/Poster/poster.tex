% !TeX program = lualatex
\documentclass[final]{beamer}
% ====================
% Packages
% ====================
\usepackage{url}
\usepackage[T1]{fontenc}
\usepackage{lmodern}
\usepackage[orientation=portrait,size=a0,scale=1.00]{beamerposter}
\usetheme{gemini}
\usecolortheme{UBI}
\usepackage{graphicx}
\usepackage[portuguese]{babel}
\usepackage{pgfplots}
\pgfplotsset{compat=1.14}
\usepackage{anyfontsize}
\usepackage{xcolor}
\usepackage{adjustbox}
\usepackage{siunitx}
\usepackage{caption}
\usepackage{tikz}
\usepackage[most]{tcolorbox}
\definecolor{p1}{HTML}{caf0f8}
\definecolor{p2}{HTML}{ade8f4}
\definecolor{p3}{HTML}{90e0ef}
\definecolor{p4}{HTML}{48cae4}
\definecolor{p5}{HTML}{00b4d8}
\definecolor{p6}{HTML}{0096c7}
\definecolor{p7}{HTML}{0077b6}
\definecolor{p8}{HTML}{023e8a}
\definecolor{p9}{HTML}{03045e}
\definecolor{myred}{HTML}{f44336}
\definecolor{mypink}{HTML}{e81e63}
\definecolor{mypurple}{HTML}{9c27b0}
\definecolor{mydeeppurple}{HTML}{673ab7}
\definecolor{myindigo}{HTML}{3f51b5}
\definecolor{myblue}{HTML}{2196f3}
\definecolor{mylightblue}{HTML}{03a9f4}
\definecolor{mycyan}{HTML}{00bcd4}
\definecolor{myteal}{HTML}{009688}
\definecolor{mygreen}{HTML}{4caf50}
\definecolor{mylightgreen}{HTML}{8bc34a}
\definecolor{mylime}{HTML}{cddc39}
\definecolor{myyellow}{HTML}{ffeb3b}
\definecolor{myamber}{HTML}{ffc107}
\definecolor{myorange}{HTML}{ff9800}
\definecolor{mydeeporange}{HTML}{ff5722}
\definecolor{mybrown}{HTML}{795548}
\definecolor{mygray}{HTML}{9e9e9e}
\definecolor{mybluegray}{HTML}{607d8b}
\definecolor{pag}{HTML}{293133}
\definecolor{fundo}{RGB}{41, 49, 51}
\definecolor{UBI}{RGB}{12, 35, 64}
% ====================
% Lengths
% ====================
% If you have N columns, choose \sepwidth and \colwidth such that
% (N+1)*\sepwidth + N*\colwidth = \paperwidth
\newlength{\sepwidth}
\newlength{\colwidth}
\setlength{\sepwidth}{0.025\paperwidth}
\setlength{\colwidth}{0.45\paperwidth}
\newcommand{\separatorcolumn}{\begin{column}{\sepwidth}\end{column}}
% ====================
% Title
% ====================
\title{Queda de uma mola ideal suspensa com massas distribuídas regularmente}
\author{José Amoreira \inst{1,2,3} \and João Santos \inst{2} \and João Esteves \inst{2}}
\institute[]{\inst{1} Laboratório de Instrumentação e Física Experimental de Partículas \and  \inst{2}Universidade da Beira Interior  \samelineand \inst{3} Centro de Matemática e Aplicações}
% ====================
% Footer
% ====================
\footercontent{
  Física 2024 \hfill
  UBI - Departamento de Física}
% ====================
% Logos
% ====================
% use this to include logos on the left and/or right side of the header:
%\logoright{\includegraphics[scale=1.5]{logos/DF2.png}}
\logoleft{\includegraphics[scale=1.5]{logos/DF2.png}}
% ====================
% Corpo
% ====================
\begin{document}
\begin{frame}[t]
\begin{columns}[t]
\separatorcolumn
\begin{column}{\colwidth}
% -------------------------------
% Secção: Motivação/Introdução
% -------------------------------
\begin{exampleblock}{Motivação/Introdução}
	Vários vídeos disponíveis na plataforma YouTube mostram a queda de uma mola
	elástica a partir de uma situação de repouso estático em que ela se encontra na
	vertical, suspensa de uma das suas extremidades. Estes vídeos são interessantes
	poque mostram a extremidade inferior da mola como que a aguardar que a
	extremidade superior a atinja, antes de começar o seu movimento de queda
	propriamente dito. 
	
  A explicação deste comportamento é dada pela teoria da elasticidade de um meio
  contínuo. A onda de deformação gerada na extremidade superior da mola quando é
  solta propaga-se longitudinalmente com uma velocidade finita, e só quando
  atinge a extremidade inferior, alterando aí o estado de deformação inicial, se
  modifica o equilíbrio de forças (peso e força elástica) que mantinham esta
  extremidade em repouso.
	
  Claramente, o modelo elementar de mola ideal, em que se despreza a sua massa,
  é insuficiente para enquadrar esta explicação, uma vez que não tendo massa,
  (1)~a mola não fica sujeita à gravidade, ou seja, não cai e (2)~a sua
  deformação é sempre uniforme, pelo que a força sobre a extremidade inferior
  altera-se instantaneamente assim que a extremidade superior inicia a sua
  queda. No entanto, será que um sistema de massas pontuais ligadas por molas
  ideiais 
	
	Com este trabalho, tentou-se descrever o movimento de queda das molas reais
	usando o modelo de mola ideal, com massas iguais distribuídas uniformemente
	sobre o seu comprimento.
\end{exampleblock}
% -------------------------------
% Secção: Formalismo
% -------------------------------
\begin{block}{Formalismo}
	\vspace{1cm}
	\newcommand{\mola}[4]%Quanto menor for o valor 3,4 e 5 dado à função mais comprimida estará a mola%
{
%------------------------------------------------------------Pontos Importantes------------------------------------------------------------%
\coordinate (bola1) at (0,{#1});
\coordinate (bola2) at (0,{#2}); 
\coordinate (bola3) at (0,-4);
\coordinate (bola4) at (0,-8);
\coordinate (reticencias) at (0,-6);
%-------------------------------------------------------------Desenhar as Molas------------------------------------------------------------%
\tikzstyle{spring}=[thick,decorate,decoration={aspect=.5, segment length=#3, amplitude=2.5mm,coil}]
\draw [spring] (bola2) -- (bola1);
\draw [spring] (bola3) -- (bola2);
\tikzstyle{spring}=[thick,decorate,decoration={aspect=.5, segment length=#4, amplitude=2.5mm,coil}]
\draw [spring] (bola4) -- (bola3);
%-------------------------------------------------------------Desenhar as Bolas------------------------------------------------------------%
\shade [ball color =white] (bola1) circle (.5) node[draw=none,inner sep = 0,scale=2,text=black]{};
\shade [ball color =white] (bola2) circle (.5) node[draw=none,inner sep = 0,scale=2,text=black]{};
\shade [ball color =white] (bola3) circle (.5) node[draw=none,inner sep = 0,scale=2,text=black]{};
\shade [ball color =white] (bola4) circle (.5) node[draw=none,inner sep = 0,scale=2,text=black]{};
%----------------------------------------------------------Desenhar as Reticências--------------------------------------------------------%
\node[rectangle,fill=white, minimum width = 2cm, minimum height = 1.1cm] (r) at (reticencias) {};
\draw (reticencias) node[ultra thick,scale=1,yshift=.05cm]{$\rotatebox{90}{$\cdots$}$} ;
}
\begin{minipage}{0.22\colwidth}
\resizebox{.2\colwidth}{!}{
\begin{tikzpicture}[scale=1, every node/.style={scale=.6}]
\mola{3}{-1.5}{3mm}{3mm}
\node[anchor=west] at (-1.5,3) {\textbf{\boldmath$1$}};
\node[anchor=west] at (-1.5,-1.5) {\textbf{\boldmath$2$}};
\node[anchor=west] at (-1.5,-4) {\textbf{\boldmath$3$}};
\node[anchor=west] at (-1.5,-8) {\textbf{\boldmath$N$}};
\draw[thick,->] (2,-8) -- (2,-6) node[draw=none,inner sep = 0,scale=1.5,xshift=.4cm,text=black]{$y$};
\end{tikzpicture}}
\end{minipage}
\begin{minipage}{0.7\colwidth}
Descrevemos uma mola ideal com constante elástica $K$, comprimento natural $L$ e massa $M$ como um sistema de $N$ massas $m=M/N$ ligadas sequencialmente por $N-1$ molas ideais com comprimento natural $l=L/(N-1)$ e constante elástica $k=K(N-1)$.\\[1ex]   
%\begin{align*}
%  l&=L/(N-1)&&\rightarrow\text{Comprimento de cada segmento}\\
%  k&=K(N-1)&& \rightarrow\text{Constante elástica de cada segmento}\\
%  m&=M/N    &&\rightarrow\text{Massa de cada partícula}
%\end{align*}
Em equilíbrio estático, as posições das várias massas são dadas por:
\begin{equation*}
  %y_i^0=y_0^0-i\left[l+\left(N-\frac{i+1}{2}\right)\frac{mg}{k}\right],\quad i>0
  y_i^0=y_1^0-(i-1)\left[l+\left(N-\frac{i}{2}\right)\frac{mg}{k}\right],
  \quad i>1
\end{equation*}
A partir do instante em que se solta a massa na extremidade superior da mola, as acelerações das várias massas são dadas por ($x_i(t)=y_i(t)-y_i^0$, $\ddot x_i=\ddot y_i$, $\omega^2=k/m$):
  \begin{align*}
    %\ddot x_0 &=-Ng+\omega^2(x_1-x_0)\\
    \ddot x_1 &=-Ng-\omega^2(x_1-x_2)\\
    \ddot x_i &= \omega^2(x_{i-1}-2x_i+x_{i+1}),\qquad i=2, \ldots, N-1\\
    %\ddot x_{N-1} &=\omega^2(x_{N-2}-x_{N-1})
    \ddot x_{N} &=\omega^2(x_{N-1}-x_{N})
  \end{align*}
\end{minipage}\\[1cm]
Este sistema de equações, com o estado inicial $x_i=0$, $\dot x_i=0$, foi resolvido em Python/Numpy \cite{harris2020array}, usando a função
\texttt{solve\_ivp} da biblioteca SciPy \cite{2020SciPy-NMeth}.

\end{block}
% -------------------------------
% Secção: Procedimento experimental
% -------------------------------
\begin{block}{Procedimento experimental}
	Para verificar a validade das soluções obtidas numericamente, estudou-se o movimento de queda de uma mola de aço (slinky) com bolas de ténis dispostas regularmente ao longo do seu comprimento. Esse movimento foi filmado a 120 fps e a posição das bolas foi obtida fotograma a fotograma usando o programa Tracker \cite{Tracker}. Verificou-se um bom acordo entre os resultados númericos e os valores observados deste modo 

\end{block}
\end{column}
\separatorcolumn
\begin{column}{\colwidth}
% -------------------------------
% Secção: Resultados
% -------------------------------
\begin{block}{Resultados}
	Na Figura A estão representados gráficos da posição calculada e observada de
	cada massas como funções do tempo, para o caso $N=3$. É patente o bom acordo
	entre os valores calculados e observados. As discrepâncias mais significativas
	ocorrem no final do intervalo de tempo representado, quando as espiras da mola
	se comprimem umas contra as outras, devido ao movimento de queda mais rápido das
	massas mais acima, efeito que altera a dinâmica puramente elástica e gravítica
	do modelo numérico.
	
	\vspace{1cm}
	\begin{figure}
		\begin{tikzpicture}
			\begin{axis}[
			%height=,
			width=.4\colwidth,
			domain=0:0.5,
			xmin=0,
			xmax=0.38,
			legend style={fill=white,draw=none},
			legend cell align={left},
			%legend pos=outer north east,
			xlabel=\textcolor{UBI}{tempo $(\unit{\second})$},
			ytick={0,0.5,1,1.5,2,2.5},yticklabels={0,0.5,1,1.5,2,2.5},
			xtick={0,0.1,0.2,0.3,0.4},xticklabels={0,0.1,0.2,0.3,0.4},
			ymax=2.57,
			ymin=0,
			ylabel=\textcolor{UBI}{Posição $(\unit{\metre})$},
			%grid=both,
			%major grid style={line width=.2pt,draw=pag!95},
			samples=400,
			%ticks=none,
			%ylabel=,
			every axis/.append style={axis line style={UBI}, tick style={UBI}}
			]	
			\addplot[mylime, smooth, thick] {+2.4508266637798486+1.92551978e-06-7.08288771e-04*x-1.46868493e+01*x^2-4.09976421e-01*x^3+2.18585827e+01*x^4-9.39435949e+00*x^5-6.34499469e+00*x^6};
			\addlegendentry[UBI]{massa 0}	
			\addplot[mypurple, smooth, thick] {0.9655805946162642+5.849073753628633*x^6+16.602620664621227*x^5-23.992790391059497*x^4+0.721408292027056*x^3-0.049464418381870844*x^2+0.0012431133324992847*x-3.3748416628826994e-06};
			\addlegendentry[UBI]{massa 1}
			\addplot[p6, smooth, thick] {0.12706810953816955+0.49592093161308015*x^6-7.2082611722205305*x^5+2.1342076920531725*x^4-0.311431870599336*x^3+0.02131369465598449*x^2-0.0005348245618400737*x+1.449321886986526e-06};
			\addlegendentry[UBI]{massa 2}	
			%%%%%%%%%%%%%%%%%%%%%%%%%%%%%%%%Laboratoriais%%%%%%%%%%%%%%%%%%%%%%%%%%%%%%%%%%%%%%%%%%%%%%%%%%%%
			\addplot[mylightgreen, loosely dotted, line width=1mm] 						{2.45082666-5.52077707e-01*(x+0.05)+3.43475522e+01*(x+0.05)^2-7.74480255e+02*(x+0.05)^3+7.67559430e+03*(x+0.05)^4-4.78466157e+04*(x+0.05)^5+1.85156584e+05*(x+0.05)^6-4.25893494e+05*(x+0.05)^7+5.29402858e+05*(x+0.05)^8-2.72302397e+05*(x+0.05)^9};
			\addplot[mydeeppurple,loosely dotted, line width=1mm] {+236878.24663477603*(x+0.05)^9-445940.6935199308*(x+0.05)^8+341684.85624595085*(x+0.05)^7-137516.49356955112*(x+0.05)^6+31308.952389044767*(x+0.05)^5-4064.7449218401653*(x+0.05)^4+286.3968820830498*(x+0.05)^3-9.702864113503072*(x+0.05)^2+0.14861681786073874*(x+0.05)+0.9655805946162642};
			\addplot[p5, loosely dotted, line width=1mm] {+5919.067897122547*(x+0.05)^9-11617.379837153167*(x+0.05)^8+9381.76346394042*(x+0.05)^7-4036.0015719443286*(x+0.05)^6+992.5740054163633*(x+0.05)^5-138.15718216106012*(x+0.05)^4+9.591644806308794*(x+0.05)^3-0.1774404113040594*(x+0.05)^2+0.019261796302454434*(x+0.05)+0.12706810953816955};        
			\end{axis}
		\end{tikzpicture}
		\begin{tikzpicture}
			\begin{axis}[
			                %height=,
			                width=.4\colwidth,
			                % domain=52:60,
			                xmin=0,xmax=.4,
			                xlabel=\textcolor{UBI}{tempo $(\unit{\second})$},
			               % ticks=none,
			                xtick={0,0.1,0.2,0.3,0.4},xticklabels={0,0.1,0.2,0.3,0.4},
			                ymin=-0.05,ymax=0.005,
			                scaled ticks = false,
			                %yticklabel style={/pgf/number format/.cd,fixed zerofill,precision=2},
			                ylabel=\textcolor{UBI}{Posição $(\unit{\centi\metre})$},
		 ytick={-0.05,-0.0225,0.005},yticklabels={-5,-2.25,0.5},
			                %grid=both,major grid style={line width=.2pt,draw=#color},
			                %samples=#,
			                %every axis/.append style={axis line style={color}, tick style={color}},
			                legend style={draw=none},
			                legend cell align={left},
			                legend pos= south west,
			                every axis/.append style={axis line style={UBI}, tick style={UBI}}
			]
			\addplot[p5, thick] table[x = T, y = X1] {1.dat};
			\addlegendentry[UBI]{$N=2$};
			\addplot[p6, thick] table[x = T, y = X2] {1.dat};
			\addlegendentry[UBI]{$N=3$};
			\addplot[p7, thick] table[x = T, y = X3] {1.dat};
			\addlegendentry[UBI]{$N=4$};
			\addplot[p8, thick] table[x = T, y = X4] {1.dat};
			\addlegendentry[UBI]{$N=5$};
			\end{axis}
		\end{tikzpicture}
		\caption{Deslocamentos a partir das posições de equilíbrio na queda de uma mola com 3 massas, a tracejado estão as curvas observadas e a contínuo estão as soluções numéricas.}
	\end{figure}
	Na Figura B está representada a posição da massa na extremidade inferior da
	mola, como função do tempo, para diferentes números (2, 3, 4 e 5) de massas
	dispostas na mola. Nota-se quanto maior for o número de massas, maior é o
	atraso no início do movimento da última massa. Este facto sugere que o
	comportamento de queda das molas demonstrado nos vídeos referidos resulta da
	densidade de massa finita das molas reais.
	\begin{center}
	\begin{tikzpicture}
			\begin{axis}[
			                %height=,
			                width=.4\colwidth,
			                % domain=52:60,
			                xmin=0,xmax=.4,
			                xlabel=\textcolor{UBI}{tempo $(\unit{\second})$},
			               % ticks=none,
			                xtick={0,0.1,0.2,0.3,0.4},xticklabels={0,0.1,0.2,0.3,0.4},
			                ymin=-0.05,ymax=0.005,
			                scaled ticks = false,
			                %yticklabel style={/pgf/number format/.cd,fixed zerofill,precision=2},
			                ylabel=\textcolor{UBI}{Posição $(\unit{\centi\metre})$},
		 ytick={-0.05,-0.0225,0.005},yticklabels={-5,-2.25,0.5},
			                %grid=both,major grid style={line width=.2pt,draw=#color},
			                %samples=#,
			                %every axis/.append style={axis line style={color}, tick style={color}},
			                legend style={draw=none},
			                legend cell align={left},
			                legend pos= south west,
			                every axis/.append style={axis line style={UBI}, tick style={UBI}}
			]
			\addplot[p5, thick] table[x = T, y = X1] {1.dat};
			\addlegendentry[UBI]{$N=2$};
			\addplot[p6, thick] table[x = T, y = X2] {1.dat};
			\addlegendentry[UBI]{$N=3$};
			\addplot[p7, thick] table[x = T, y = X3] {1.dat};
			\addlegendentry[UBI]{$N=4$};
			\addplot[p8, thick] table[x = T, y = X4] {1.dat};
			\addlegendentry[UBI]{$N=5$};
			\end{axis}
		\end{tikzpicture}
	\end{center}
\end{block}
% -------------------------------
% Secção: Conclusões
% -------------------------------
\begin{exampleblock}{Conclusões}
    \begin{itemize}
      \item Sed et augue accumsan nibh ullamcorper accumsanam dictum urna tortor, ut pretium leo eleifend.  
      \item Donec suscipit, urna quis tempus consectetur, quam est placerat ante, et scelerisque metus velit. 
      \item Nam dictum urna tortor, ut pretium leo eleifend efficitur.
      \item Praesent blandit faucibus quam, et tincidunt mauris sagittis eget 2.2\%.
      \item Dolor sit amet, consectetur adipiscing elit. Mauris in nulla ultricies suscipit.
    \end{itemize}
\end{exampleblock}
% -------------------------------
% Secção: Referências
% -------------------------------
\begin{block}{Referências}
    \nocite{*}
    \footnotesize{\bibliographystyle{plainurl}\bibliography{poster}}
\end{block}
% -------------------------------
% Section: Portfolio
% -------------------------------
\begin{block}{Mais Informações}
    \begin{figure}[h]
    \centering
    \includegraphics[height=8cm]{images/qrcode2.png}
    \label{fig:figure3}
\end{figure}
\end{block}
\end{column}
\separatorcolumn
\end{columns}
\end{frame}
\end{document}
