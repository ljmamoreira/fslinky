% !TeX program = lualatex
\documentclass[final]{beamer}
% ====================
% Packages
% ====================
\usepackage{url}
\usepackage[T1]{fontenc}
\usepackage{lmodern}
\usepackage[orientation=portrait,size=a0,scale=1.00]{beamerposter}
\usetheme{gemini}
\usecolortheme{UBI}
\usepackage{graphicx}
\usepackage{booktabs}
\usepackage{tikz}
\usepackage[portuguese]{babel}
\usepackage{pgfplots}
\usepackage{subfigure}
\usepackage{float}
\usepackage{placeins}
\pgfplotsset{compat=1.14}
\usepackage{anyfontsize}
\usepackage{xcolor}
\usepackage[skip=2pt,font=normalsize]{subcaption}
\usepackage{adjustbox}
\usepackage{tikz}
\usetikzlibrary{decorations.pathmorphing,patterns}
\usetikzlibrary{decorations.pathmorphing,patterns}
\usetikzlibrary{calc,patterns,decorations.markings}
\usetikzlibrary{positioning}
\usepackage[most]{tcolorbox}
\definecolor{p1}{HTML}{caf0f8}
\definecolor{p2}{HTML}{ade8f4}
\definecolor{p3}{HTML}{90e0ef}
\definecolor{p4}{HTML}{48cae4}
\definecolor{p5}{HTML}{00b4d8}
\definecolor{p6}{HTML}{0096c7}
\definecolor{p7}{HTML}{0077b6}
\definecolor{p8}{HTML}{023e8a}
\definecolor{p9}{HTML}{03045e}
\usepackage{colortbl}
\usepackage{tkz-euclide}
\colorlet{xcol}{blue!60!black}
\usepackage{capt-of}
\definecolor{myred}{HTML}{f44336}
\definecolor{mypink}{HTML}{e81e63}
\definecolor{mypurple}{HTML}{9c27b0}
\definecolor{mydeeppurple}{HTML}{673ab7}
\definecolor{myindigo}{HTML}{3f51b5}
\definecolor{myblue}{HTML}{2196f3}
\definecolor{mylightblue}{HTML}{03a9f4}
\definecolor{mycyan}{HTML}{00bcd4}
\definecolor{myteal}{HTML}{009688}
\definecolor{mygreen}{HTML}{4caf50}
\definecolor{mylightgreen}{HTML}{8bc34a}
\definecolor{mylime}{HTML}{cddc39}
\definecolor{myyellow}{HTML}{ffeb3b}
\definecolor{myamber}{HTML}{ffc107}
\definecolor{myorange}{HTML}{ff9800}
\definecolor{mydeeporange}{HTML}{ff5722}
\definecolor{mybrown}{HTML}{795548}
\definecolor{mygray}{HTML}{9e9e9e}
\definecolor{mybluegray}{HTML}{607d8b}
\definecolor{pag}{HTML}{293133}
\definecolor{fundo}{RGB}{41, 49, 51}
\definecolor{UBI}{RGB}{12, 35, 64}
\usepackage{siunitx}
% ====================
% Lengths
% ====================
% If you have N columns, choose \sepwidth and \colwidth such that
% (N+1)*\sepwidth + N*\colwidth = \paperwidth
\newlength{\sepwidth}
\newlength{\colwidth}
\setlength{\sepwidth}{0.025\paperwidth}
\setlength{\colwidth}{0.45\paperwidth}
\newcommand{\separatorcolumn}{\begin{column}{\sepwidth}\end{column}}
% ====================
% Title
% ====================
\title{Queda de uma mola ideal suspensa com massas distribuídas regularmente}
\author{José Amoreira \inst{1,2,3} \and João Santos \inst{2} \and João Esteves \inst{2}}
\institute[]{\inst{1} Laboratório de Instrumentação e Física Experimental de Partículas \and  \inst{2}Universidade da Beira Interior  \samelineand \inst{3} Centro de Matemática e Aplicações}
% ====================
% Footer (optional)
% ====================
\footercontent{
  Física 2024 \hfill
  UBI - Departamento de Física}
% (can be left out to remove footer)
% ====================
% Logo (optional)
% ====================
% use this to include logos on the left and/or right side of the header:
%\logoright{\includegraphics[scale=1.5]{logos/DF2.png}}
\logoleft{\includegraphics[scale=1.5]{logos/DF2.png}}
% ====================
% Body
% ====================
\begin{document}

\begin{frame}[t]
\begin{columns}[t]
\separatorcolumn

\begin{column}{\colwidth}

% ----------------------------------
% Abstract
% ----------------------------------
  \begin{exampleblock}{Motivação/Introdução}
Vários vídeos disponíveis na plataforma YouTube mostram a queda de uma mola
elástica a partir de uma situação de repouso estático em que ela se encontra na
vertical, suspensa de uma das suas extremidades. Estes vídeos são interessantes
poque mostram a extremidade inferior da mola como que a aguardar que a
extremidade superior a atinja, antes de começar o seu movimento de queda
propriamente dito. 

A explicação deste comportamento é dada pela teoria da elasticidade de um meio
contínuo. A onda de deformação gerada na extremidade superior da mola quando é
solta propaga-se longitudinalmente com uma velocidade finita, e só quando atinge
a extremidade inferior, alterando aí o estado de deformação inicial, se altera o
equilíbrio de forças (peso e força elástica) que mantinham esta extremidade em
repouso.

Claramente, o modelo elementar de mola ideal, em que se despreza a sua massa, é
insuficiente para enquadrar esta explicação, uma vez que não tendo massa, (1)~a
mola não fica sujeita à gravidade, ou seja, não cai e (2)~a sua deformação é
sempre uniforme, pelo que a força sobre a extremidade inferior altera-se
instantaneamente assim que a extremidade superior inicia a sua queda. 

Com este trabalho, tentou-se descrever o movimento de queda das molas reais
usando o modelo de mola ideal, com massas iguais distribuídas uniformemente
sobre o seu comprimento.
  \end{exampleblock}
  

\begin{block}{Formalismo}
\vspace{1cm}
\newcommand{\mola}[4]%Quanto menor for o valor 3,4 e 5 dado à função mais comprimida estará a mola%
{
%------------------------------------------------------------Pontos Importantes------------------------------------------------------------%
\coordinate (bola1) at (0,{#1});
\coordinate (bola2) at (0,{#2}); 
\coordinate (bola3) at (0,-4);
\coordinate (bola4) at (0,-8);
\coordinate (reticencias) at (0,-6);
%-------------------------------------------------------------Desenhar as Molas------------------------------------------------------------%
\tikzstyle{spring}=[thick,decorate,decoration={aspect=.5, segment length=#3, amplitude=2.5mm,coil}]
\draw [spring] (bola2) -- (bola1);
\draw [spring] (bola3) -- (bola2);
\tikzstyle{spring}=[thick,decorate,decoration={aspect=.5, segment length=#4, amplitude=2.5mm,coil}]
\draw [spring] (bola4) -- (bola3);
%-------------------------------------------------------------Desenhar as Bolas------------------------------------------------------------%
\shade [ball color =white] (bola1) circle (.5) node[draw=none,inner sep = 0,scale=2,text=black]{};
\shade [ball color =white] (bola2) circle (.5) node[draw=none,inner sep = 0,scale=2,text=black]{};
\shade [ball color =white] (bola3) circle (.5) node[draw=none,inner sep = 0,scale=2,text=black]{};
\shade [ball color =white] (bola4) circle (.5) node[draw=none,inner sep = 0,scale=2,text=black]{};
%----------------------------------------------------------Desenhar as Reticências--------------------------------------------------------%
\node[rectangle,fill=white, minimum width = 2cm, minimum height = 1.1cm] (r) at (reticencias) {};
\draw (reticencias) node[ultra thick,scale=1,yshift=.05cm]{$\rotatebox{90}{$\cdots$}$} ;
}
\begin{minipage}{0.22\colwidth}
\resizebox{.2\colwidth}{!}{
\begin{tikzpicture}[scale=1, every node/.style={scale=.6}]
\mola{3}{-1.5}{3mm}{3mm}
\node[anchor=west] at (-1.5,3) {\textbf{\boldmath$1$}};
\node[anchor=west] at (-1.5,-1.5) {\textbf{\boldmath$2$}};
\node[anchor=west] at (-1.5,-4) {\textbf{\boldmath$3$}};
\node[anchor=west] at (-1.5,-8) {\textbf{\boldmath$N$}};
\draw[thick,->] (2,-8) -- (2,-6) node[draw=none,inner sep = 0,scale=1.5,xshift=.4cm,text=black]{$y$};
\end{tikzpicture}}
\end{minipage}
\begin{minipage}{0.7\colwidth}
Descrevemos uma mola ideal com constante elástica $K$, comprimento natural $L$ e massa $M$ como um sistema de $N$ massas $m=M/N$ ligadas sequencialmente por $N-1$ molas ideais com comprimento natural $l=L/(N-1)$ e constante elástica $k=K(N-1)$.\\[1ex]   
%\begin{align*}
%  l&=L/(N-1)&&\rightarrow\text{Comprimento de cada segmento}\\
%  k&=K(N-1)&& \rightarrow\text{Constante elástica de cada segmento}\\
%  m&=M/N    &&\rightarrow\text{Massa de cada partícula}
%\end{align*}
Em equilíbrio estático, as posições das várias massas são dadas por:
\begin{equation*}
  %y_i^0=y_0^0-i\left[l+\left(N-\frac{i+1}{2}\right)\frac{mg}{k}\right],\quad i>0
  y_i^0=y_1^0-(i-1)\left[l+\left(N-\frac{i}{2}\right)\frac{mg}{k}\right],
  \quad i>1
\end{equation*}
A partir do instante em que se solta a massa na extremidade superior da mola, as acelerações das várias massas são dadas por ($x_i(t)=y_i(t)-y_i^0$, $\ddot x_i=\ddot y_i$, $\omega^2=k/m$):
  \begin{align*}
    %\ddot x_0 &=-Ng+\omega^2(x_1-x_0)\\
    \ddot x_1 &=-Ng-\omega^2(x_1-x_2)\\
    \ddot x_i &= \omega^2(x_{i-1}-2x_i+x_{i+1}),\qquad i=2, \ldots, N-1\\
    %\ddot x_{N-1} &=\omega^2(x_{N-2}-x_{N-1})
    \ddot x_{N} &=\omega^2(x_{N-1}-x_{N})
  \end{align*}
\end{minipage}\\[1cm]
Este sistema de equações, com o estado inicial $x_i=0$, $\dot x_i=0$, foi resolvido em Python/Numpy \cite{harris2020array}, usando a função
\texttt{solve\_ivp} da biblioteca SciPy \cite{2020SciPy-NMeth}.

\end{block}

\begin{block}{Procedimento experimental}
Para verificar a validade das soluções obtidas numericamente, estudou-se o movimento de queda de uma mola de aço (slinky) com bolas de ténis dispostas regularmente na mola. Esse movimento foi filmado a 120 fps e a posição das bolas foi obtida fotograma a fotograma usando o programa Tracker \cite{Tracker}. Verificou-se um bom acordo entre os resultados númericos e os valores observados deste modo 


%   Foi observada e gravada a queda do conjunto mola e bola para compreender o seu comportamento. O experimento foi realizado de duas maneiras diferentes, com duas bolas e uma mola, e com três bolas, dividindo a mola ao meio. Ambas as execuções foram gravadas em câmara lenta, a uma taxa de 120 frames por segundo, e posteriormente analisadas usando o software Tracker.
%    % This flow chart is created by the author

% adjustbox is used to limit the figure inside the page
% -- means normal arrow
%  -| horizontal followed by the vertical arrow
%  |- vertical followed by the horizontal arrow


\begin{figure}

    \begin{center}
        \begin{adjustbox}{max height=\textheight, center, width=0.8\textwidth}
            \begin{tikzpicture}
\tikzstyle{spring}=[thick,decorate,decoration={aspect=.5, segment length=5, amplitude=2.5mm,coil, post length=0, pre length=0}]
   % \draw[pattern={north east lines}] (-1,5.2) rectangle (1,5.4);
    \draw [spring] (0,0) -- (0,-5);
    \node[] at (0,-8) {$(a)$};
    \draw [fill=white] (0,0) circle (.5) node[draw=none,inner sep = 0,scale=2,text=black]{$m$};
    \draw [fill=white] (0,-5) circle (.5) node[draw=none,inner sep = 0,scale=2,text=black]{$m$};
%________________________________________________________________%
\tikzstyle{spring}=[thick,decorate,decoration={aspect=.5, segment length=6, amplitude=2.5mm,coil, post length=0, pre length=0}]
    \draw [spring] (3,0) -- (3,-3);
   \draw [fill=white] (3,0) circle (.5) node[draw=none,inner sep = 0,scale=2,text=black]{$m$};
    \tikzstyle{spring}=[thick,decorate,decoration={aspect=.5, segment length=6, amplitude=2.5mm,coil, post length=0, pre length=0}]
    \draw [spring] (3,-3) -- (3,-7);
    \draw [fill=white] (3,-3) circle (.5) node[draw=none,inner sep = 0,scale=2,text=black]{$m$};
    \draw [fill=white] (3,-7) circle (.5) node[draw=none,inner sep = 0,scale=2,text=black]{$m$};
    \node[] at (3,-8) {$(b)$};
            \end{tikzpicture}
        \end{adjustbox}
    \end{center}
    \caption{Caption for flowchart.}
    \label{fig:figure1}
\end{figure}
% A constante elástica tanto para a mola completa quanto para a metade da mola foi determinada por meio de outro experimento. Com esses resultados em mãos, os valores teóricos foram calculados utilizando o Python e comparados com os resultados experimentais obtidos pelo Tracker. \emph{(mais informação no código QR)}
\end{block}

% -------------------------------
% Section: Descriptive Statistics
% -------------------------------


\end{column}

\separatorcolumn

\begin{column}{\colwidth}

% -------------------------------
% Section: Results and discussion
% -------------------------------

\begin{block}{Resultados}
Na figura estão representados gráficos da posição calculada e observada de cada massa como funções do tempo.
\vspace{1cm}
   \begin{figure}
   \centering
  \begin{tikzpicture}
		\begin{axis}[
			%height=,
			width=.6\colwidth,
			domain=0:0.5,
			xmin=0,
			xmax=0.38,
               		legend style={fill=white,draw=none},
               		legend cell align={left},
               		%legend pos=outer north east,
			xlabel=\textcolor{UBI}{tempo $(\unit{\second})$},
			ytick={},
			yticklabels={},
			xtick={},
			xticklabels={},
			ymax=2.57,
			ymin=0,
			ylabel=\textcolor{UBI}{Posição $(\unit{\metre})$},
			%grid=both,
			major grid style={line width=.2pt,draw=pag!95},
			samples=400,
			ticks=none,
			%ylabel=,
			every axis/.append style={axis line style={UBI}, tick style={pag!80}}
			]
			
			\addplot[mylime, smooth, thick] {+2.4508266637798486+1.92551978e-06-7.08288771e-04*x-1.46868493e+01*x^2-4.09976421e-01*x^3+2.18585827e+01*x^4-9.39435949e+00*x^5-6.34499469e+00*x^6};
	\addlegendentry[UBI]{massa 0}	
			\addplot[mypurple, smooth, thick] {0.9655805946162642+5.849073753628633*x^6+16.602620664621227*x^5-23.992790391059497*x^4+0.721408292027056*x^3-0.049464418381870844*x^2+0.0012431133324992847*x-3.3748416628826994e-06};
           \addlegendentry[UBI]{massa 1}
			\addplot[p6, smooth, thick] {0.12706810953816955+0.49592093161308015*x^6-7.2082611722205305*x^5+2.1342076920531725*x^4-0.311431870599336*x^3+0.02131369465598449*x^2-0.0005348245618400737*x+1.449321886986526e-06};
           \addlegendentry[UBI]{massa 2}	
			
			\addplot[mylightgreen, loosely dotted, line width=1mm] {2.45082666-5.52077707e-01*(x+0.05)+3.43475522e+01*(x+0.05)^2-7.74480255e+02*(x+0.05)^3+7.67559430e+03*(x+0.05)^4-4.78466157e+04*(x+0.05)^5+1.85156584e+05*(x+0.05)^6-4.25893494e+05*(x+0.05)^7+5.29402858e+05*(x+0.05)^8-2.72302397e+05*(x+0.05)^9};%lab
			
           
           
           
            %\addlegendentry[UBI]{Laboratorial (bola 1)}
			
			\addplot[mydeeppurple, dotted, thick] {+236878.24663477603*(x+0.05)^9-445940.6935199308*(x+0.05)^8+341684.85624595085*(x+0.05)^7-137516.49356955112*(x+0.05)^6+31308.952389044767*(x+0.05)^5-4064.7449218401653*(x+0.05)^4+286.3968820830498*(x+0.05)^3-9.702864113503072*(x+0.05)^2+0.14861681786073874*(x+0.05)+0.9655805946162642};%lab
          % \addlegendentry[UBI]{Laboratorial (bola 2)}
			
			\addplot[p5, dotted, thick] {+5919.067897122547*(x+0.05)^9-11617.379837153167*(x+0.05)^8+9381.76346394042*(x+0.05)^7-4036.0015719443286*(x+0.05)^6+992.5740054163633*(x+0.05)^5-138.15718216106012*(x+0.05)^4+9.591644806308794*(x+0.05)^3-0.1774404113040594*(x+0.05)^2+0.019261796302454434*(x+0.05)+0.12706810953816955};        
		\end{axis}
	\end{tikzpicture}
	\caption{Deslocamentos a partir das posições de equilíbrio na queda de uma mola com 3 massas, a tracejado estão as curvas observadas e a contínuo estão as soluções numéricas.}
   \end{figure}
\end{block}

% -------------------------------
% Section: Conclusions
% -------------------------------
   \begin{exampleblock}{Conclusões}
    \begin{itemize}
      \item Sed et augue accumsan nibh ullamcorper accumsanam dictum urna tortor, ut pretium leo eleifend.  
      \item Donec suscipit, urna quis tempus consectetur, quam est placerat ante, et scelerisque metus velit. 
      \item Nam dictum urna tortor, ut pretium leo eleifend efficitur.
      \item Praesent blandit faucibus quam, et tincidunt mauris sagittis eget 2.2\%.
      \item Dolor sit amet, consectetur adipiscing elit. Mauris in nulla ultricies suscipit.
    \end{itemize}
  \end{exampleblock}


  \begin{block}{Referências}

    \nocite{*}
    \footnotesize{\bibliographystyle{plainurl}\bibliography{poster}}

  \end{block}

% -------------------------------
% Section: Portfolio
% -------------------------------
  \begin{block}{Mais Informações}

    \begin{figure}[h]
    \centering
    \includegraphics[height=8cm]{images/qrcode2.png}
    \label{fig:figure3}
\end{figure}

  \end{block}

\end{column}
\separatorcolumn



\end{columns}
\end{frame}

\end{document}