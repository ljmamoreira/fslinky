%-------------------------------------
% This template comes from Anish Athalye (Unofficial University of Cambridge Poster Template). 

% The Poster Template has been modified by Dr. Rahul Raoniar to fulfill B.Tech/Master/Ph.D./PostDoc student's poster presentation requirements.

% Description: I made this unofficial Poster Template for the Indian Institute of Technology Bombay (IITB). Feel free to use it, modify it, and share it. 

% Thank you note: A huge thanks goes to Anish Athalye for the original template.
%-------------------------------------


\documentclass[final]{beamer}

% ====================
% Packages
% ====================
\usepackage{url}
\usepackage[T1]{fontenc}
\usepackage{lmodern}
\usepackage[orientation=portrait,size=a0,scale=1.0]{beamerposter}
\usetheme{gemini}
\usecolortheme{UBI}
\usepackage{graphicx}
\usepackage{booktabs}
\usepackage{tikz}
\usepackage[portuguese]{babel}
\usepackage{pgfplots}
\usepackage{subfigure}
\pgfplotsset{compat=1.14}
\usepackage{anyfontsize}
\usepackage{xcolor}
\usepackage[skip=2pt,font=normalsize]{subcaption}
\usepackage{adjustbox}

% ----------------------------------
% For plotting study methodology
% ----------------------------------

\usepackage{tikz}
\usetikzlibrary{shapes.geometric, arrows}

% Defining Tickz Style
\tikzstyle{startstop} = [rectangle, rounded corners, minimum width=3cm, minimum height=1cm, text centered, text width = 10cm, draw=black, fill=white]

% \tikzstyle{io} = [trapezium, trapezium left angle=70, trapezium right angle=110, minimum width=3cm, minimum height=1cm, text centered, text width = 4.5cm, draw=black, fill=blue!30]

\tikzstyle{process} = [rectangle, minimum width=3cm, minimum height=1cm, text centered, text width = 6cm, draw=black, fill=white, text width = 10cm]

% \tikzstyle{decision} = [diamond, minimum width=3cm, minimum height=1cm, text centered, draw=black, fill=green!30]

\tikzstyle{arrow} = [ultra thick,->,>=stealth]


% ====================
% Lengths
% ====================

% If you have N columns, choose \sepwidth and \colwidth such that
% (N+1)*\sepwidth + N*\colwidth = \paperwidth
\newlength{\sepwidth}
\newlength{\colwidth}
\setlength{\sepwidth}{0.025\paperwidth}
\setlength{\colwidth}{0.45\paperwidth}

\newcommand{\separatorcolumn}{\begin{column}{\sepwidth}\end{column}}

% ====================
% Title
% ====================

\title{Queda de uma mola ideal suspensa com massas distribuídas regularmente}

\author{José Amoreira \inst{1,2,3} \and João Santos \inst{1} \and João Esteves \inst{1}}

\institute[]{\inst{1}Universidade da Beira Interior  \samelineand \inst{2} Centro de Matemática e Aplicações \and \inst{3} Laboratório de Instrumentação e Física Experimental de Partículas}

% ====================
% Footer (optional)
% ====================

\footercontent{
  Física 2024 \hfill
  UBI - Departamento de Física}
% (can be left out to remove footer)


% ====================
% Logo (optional)
% ====================

% use this to include logos on the left and/or right side of the header:
%\logoright{\includegraphics[scale=1.5]{logos/DF2.png}}
\logoleft{\includegraphics[scale=1.5]{logos/DF2.png}}

% ====================
% Body
% ====================

\begin{document}

\begin{frame}[t]
\begin{columns}[t]
\separatorcolumn

\begin{column}{\colwidth}

% ----------------------------------
% Abstract
% ----------------------------------
  \begin{block}{Motivação}
Vários vídeos disponíveis na plataforma YouTube mostram a queda de uma mola
elástica a partir de uma situação de repouso estático em que ela se encontra na
vertical, suspensa de uma das suas extremidades. Estes vídeos são interessantes
poque mostram a extremidade inferior da mola como que a aguardar que a
extremidade superior a atinja, antes de começar o seu movimento de queda
propriamente dito.

A explicação deste comportamento é dada pela teoria da elasticidade. A onda de
deformação gerada na extremidade superior da mola quando é solta propaga-se
longitudinalmente com uma velocidade finita, e só quando atinge a extremidade
inferior, alterando aí o estado de deformação inicial, se altera o equilíbrio de
forças (peso e força elástica) que mantinham esta extremidade em repouso.
  \end{block}
  
% ----------------------------------
% Section: Literature review
% ----------------------------------
  \begin{alertblock}{Resumo}

    jmfjd fgghjgfh

    \heading{Research gaps}
    Pjdnvfj

  \end{alertblock}

\begin{block}{Descrição analítica}
    Aplicando a segunda lei de Newton a este formalismo, ficamos com:
    \begin{align*}
        \ddot x_0 &=-Ng+\omega^2(x_1-x_0)\\[1cm] 
        \ddot x_i &= \omega^2(x_{i-1}-2x_i+x_{i+1}),\qquad i=1, \ldots, N-2\\[1cm]
         \ddot x_{N-1} &=\omega^2(x_{N-2}-x_{N-1})
    \end{align*}    
\end{block}

\begin{block}{Procedimento experimental}
   Foi observada e gravada a queda do conjunto mola e bola para compreender o seu comportamento. O experimento foi realizado de duas maneiras diferentes, com duas bolas e uma mola, e com três bolas, dividindo a mola ao meio. Ambas as execuções foram gravadas em câmara lenta, a uma taxa de 120 frames por segundo, e posteriormente analisadas usando o software Tracker.
    % This flow chart is created by the author

% adjustbox is used to limit the figure inside the page
% -- means normal arrow
%  -| horizontal followed by the vertical arrow
%  |- vertical followed by the horizontal arrow


\begin{figure}

    \begin{center}
        \begin{adjustbox}{max height=\textheight, center, width=0.8\textwidth}
            \begin{tikzpicture}
\tikzstyle{spring}=[thick,decorate,decoration={aspect=.5, segment length=5, amplitude=2.5mm,coil, post length=0, pre length=0}]
   % \draw[pattern={north east lines}] (-1,5.2) rectangle (1,5.4);
    \draw [spring] (0,0) -- (0,-5);
    \node[] at (0,-8) {$(a)$};
    \draw [fill=white] (0,0) circle (.5) node[draw=none,inner sep = 0,scale=2,text=black]{$m$};
    \draw [fill=white] (0,-5) circle (.5) node[draw=none,inner sep = 0,scale=2,text=black]{$m$};
%________________________________________________________________%
\tikzstyle{spring}=[thick,decorate,decoration={aspect=.5, segment length=6, amplitude=2.5mm,coil, post length=0, pre length=0}]
    \draw [spring] (3,0) -- (3,-3);
   \draw [fill=white] (3,0) circle (.5) node[draw=none,inner sep = 0,scale=2,text=black]{$m$};
    \tikzstyle{spring}=[thick,decorate,decoration={aspect=.5, segment length=6, amplitude=2.5mm,coil, post length=0, pre length=0}]
    \draw [spring] (3,-3) -- (3,-7);
    \draw [fill=white] (3,-3) circle (.5) node[draw=none,inner sep = 0,scale=2,text=black]{$m$};
    \draw [fill=white] (3,-7) circle (.5) node[draw=none,inner sep = 0,scale=2,text=black]{$m$};
    \node[] at (3,-8) {$(b)$};
            \end{tikzpicture}
        \end{adjustbox}
    \end{center}
    \caption{Caption for flowchart.}
    \label{fig:figure1}
\end{figure}
 A constante elástica tanto para a mola completa quanto para a metade da mola foi determinada por meio de outro experimento. Com esses resultados em mãos, os valores teóricos foram calculados utilizando o Python e comparados com os resultados experimentais obtidos pelo Tracker. \emph{(mais informação no código QR)}
\end{block}

% -------------------------------
% Section: Descriptive Statistics
% -------------------------------


\end{column}

\separatorcolumn

\begin{column}{\colwidth}

% -------------------------------
% Section: Results and discussion
% -------------------------------

\begin{block}{Resultados}
    
    Após realizar o procedimento todo explicado no bloco anterior, podemos então observar os resultados:
    \begin{figure}
    \centering
    \subfigure[]{\includegraphics[width=0.3\columnwidth]{images/1.png}} 
    \subfigure[]{\includegraphics[width=0.3\columnwidth]{images/2.png}} 
    \subfigure[]{\includegraphics[width=0.3\columnwidth]{images/3.png}}
    \caption{(a) Resultados experimentais (b) Resultados teóricos (c) Comparação}
    \label{fig:foobar}
\end{figure}
    E, observando as três no mesmo gráfico
   \begin{figure}
   \includegraphics[width=.5\columnwidth]{images/4v2.jpg}
   \caption{Comparação dos resultados experimentais e numéricos.}
   \end{figure}
\end{block}

% -------------------------------
% Section: Conclusions
% -------------------------------
   \begin{block}{Conclusões}
    \begin{itemize}
      \item Sed et augue accumsan nibh ullamcorper accumsanam dictum urna tortor, ut pretium leo eleifend.  
      \item Donec suscipit, urna quis tempus consectetur, quam est placerat ante, et scelerisque metus velit. 
      \item Nam dictum urna tortor, ut pretium leo eleifend efficitur.
      \item Praesent blandit faucibus quam, et tincidunt mauris sagittis eget 2.2\%.
      \item Dolor sit amet, consectetur adipiscing elit. Mauris in nulla ultricies suscipit.
    \end{itemize}
  \end{block}


% -------------------------------
% Section: What is already known about this subject?
% -------------------------------
  \begin{exampleblock}{What is already known about this subject?}

    \begin{itemize}
      \item \textbf{Lorem ipsum dolor sit amet}, consectetur adipiscing elit. Mauris in nulla ac leo ultricies suscipit.
      \item The \textbf{Duis vestibulum augue} in leo placerat, sit amet pharetra mi elementum.
      \item \textbf{Fusce sit amet} velit pulvinar, feugiat velit sit amet, tristique dolor.
    \end{itemize}

  \end{exampleblock}

  
% -------------------------------
% Section: What does this study add?
% -------------------------------
  \begin{exampleblock}{What does this study add?}
    \begin{itemize}
      \item Lorem ipsum dolor sit amet, consectetur adipiscing elit. Mauris in nulla ac leo ultricies suscipit.
      \item The Duis vestibulum augue in leo placerat, sit amet pharetra mi elementum.
      \item Fusce sit amet velit pulvinar, feugiat velit sit amet, tristique dolor.
    \end{itemize}

  \end{exampleblock}


% -------------------------------
% Section: Practical Implications
% -------------------------------
  \begin{exampleblock}{Practical implications}
    \begin{itemize}
      \item Ipsum dolor sit amet, consectetur adipiscing elit. Mauris in nulla ultricies suscipit.
      \item Duis augue in leo placerat, sit amet pharetra mi elementum.
      \item Sit amet pulvinar, feugiat velit sit amet, tristique dolor.
    \end{itemize}

  \end{exampleblock}

% -------------------------------
% Section: References
% ------------------------------- 

  \begin{block}{Referencias}

    \nocite{*}
    \footnotesize{\bibliographystyle{plainurl}\bibliography{poster}}

  \end{block}

% -------------------------------
% Section: Portfolio
% -------------------------------
  \begin{block}{Mais Informações}

    \begin{figure}[h]
    \centering
    \includegraphics[height=8cm]{images/qrcode2.png}
    \label{fig:figure3}
\end{figure}

  \end{block}

\end{column}
\separatorcolumn



\end{columns}
\end{frame}

\end{document}
