\documentclass{article}
\usepackage[utf8]{inputenc}
\usepackage[T1]{fontenc}
\usepackage[portuguese]{babel}
\usepackage[a4paper]{geometry}
\usepackage{amsmath, icomma, tikz, physics}
\usetikzlibrary{decorations.pathmorphing,patterns}
\title{Queda de uma mola ideal suspensa com massas distribuídas regularmente}
\author{}
\date{Janeiro de 2024}
\begin{document}
\maketitle
\section{Introdução}
\begin{minipage}[t]{0.80\linewidth}
  Consideremos uma mola ideal suspensa verticalmente do teto, com comprimento
  natural $L_0$ e constante elástica $K$, com $N$ partículas pontuais iguais
  dispostas regularmente ao longo do seu comprimento, com massas $m=M/N$ com $M$
  constante, numeradas de 0 a $N-1$ (ver a figura). Seja $x_i=x_i(t)$ a posição
  da $i$-ésima massa. O sistema é equivalente a um conjunto de $N-1$ molas com
  comprimento natural $l_0=L_0/(N-1)$ e constante elásticas $k=K(N-1)$.

  Em equilíbrio estático, a resultante das forças aplicada em cada massa (a
  força elástica exercida pela mola acima dela e o peso de toda a cadeia de
  massas suspensas nessa mesma mola) deve ser nula. Assim, a condição de
  equilíbrio estático da $i$-ésima bola é
  \begin{equation}
    x^0_i = x^0_{i-1} + (N-i)\frac{mg}{k} + l_0, \qquad i\geq1.
  \end{equation}
\end{minipage}\hfill
\begin{tikzpicture}[baseline=(current bounding box.north)]
  \small
  % Supporting structure
  \fill [pattern = north west lines] (-0.75,0) rectangle ++(1.5,.2);
  \draw[thick] (-0.75,0) -- ++(1.5,0);
  \draw (0.80,0) -- (0.90,0);
  \draw [->] (0.85,0) -- +(0,-1) node[below]{$x$};
  % Spring
  \tikzset
  {
    spring/.style n args={2}{decorate,
                   decoration={coil, aspect=0.4,
                               segment length=#1,
                               amplitude = 2mm,
                               %pre length=2mm,
                               %post length=#2
                              }
                            }
  }
  \draw (0,0) -- (0,-0.4) coordinate(a);
  \draw [spring={1.5mm}{0.5mm}] ([yshift=-1.8mm]a) -- ++(0,-1.45)
    coordinate[yshift=-1.65mm](b);
  \draw [spring={1.0mm}{0mm}] ([yshift=-1.8mm]b)-- ++(0,-0.61) coordinate (c);
  \path (a) node[circle, draw, fill=white,inner sep=1]{\rule{0mm}{3mm}}
    node[left, inner sep=8]{0};
  \path (b) node[circle, draw, fill=white,inner sep=1]{\rule{0mm}{3mm}}
    node[left, inner sep=8]{1};
  %\fill (c) circle(0.1);
  \path (c) node[inner sep=0,below]{$\vdots$};
  \draw [spring={0.7mm}{1mm}] (0,-3.75) -- ++(0,-0.405)
    node[below,circle,fill=white, draw, inner sep=0]{\rule{0mm}{3mm}}
    node[shift={(-7mm,-2mm)}]{$N-1$};
\end{tikzpicture}

\vspace{0.75em}
\noindent
Sobre a massa no topo desta cadeia ($i=0$), há outra força aplicada, a que
sustem todo o sistema. Esta igualdade recursiva pode escrever-se na forma
\begin{equation}
  x^0_i = x^0_0 + i\qty[l_0 + \qty(N-\frac{i+1}{2})\frac{mg}{k}].
\end{equation}
No instante $t=0$ a 0-ésima massa é largada, e o sistema cai partindo do
repouso.

\end{document}
