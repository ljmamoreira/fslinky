\documentclass{article}
\usepackage[utf8]{inputenc}
\usepackage[T1]{fontenc}
\usepackage[portuguese]{babel}
\usepackage[a4paper]{geometry}
\usepackage{amsmath, icomma, tikz, physics}
\usetikzlibrary{decorations.pathmorphing,patterns}
\title{Queda de uma mola ideal suspensa com massas distribuídas regularmente}
\author{}
\date{Janeiro de 2024}
\begin{document}
\maketitle
\section{Introdução}
Consideremos uma mola ideal suspensa verticalmente do teto, com massas iguais
dispostas regularmente ao longo do seu comprimento.
\begin{tikzpicture}[black!75,thick]
 
% Supporting structure
\fill [pattern = north west lines] (-1.5,0) rectangle ++(3,.2);
\draw[thick] (-1.5,0) -- ++(3,0);
\draw (0,0) -- (0,-0.3);
\fill (0,-0.3) circle (0.1);
\draw
[
    decoration={
        coil,
        segment length = 2mm,
        amplitude = 2mm,
        aspect = 0.5,
        post length = 0mm,
        pre length = 0mm},
    decorate] (0,-0.3) -- ++(0,-2.5);
 
\end{tikzpicture}
\end{document}
