\documentclass{article}
\usepackage[utf8]{inputenc}
\usepackage[T1]{fontenc}
\usepackage[portuguese]{babel}
\usepackage[a4paper]{geometry}
\usepackage{amsmath, icomma, tikz, physics}
\usetikzlibrary{decorations.pathmorphing,patterns}
\title{Queda de uma mola ideal suspensa com massas distribuídas regularmente}
\author{}
\date{Janeiro de 2024}
\begin{document}
\maketitle
\section{Introdução}
\begin{minipage}[t]{0.85\linewidth}
  Consideremos uma mola ideal suspensa verticalmente do teto, com $N$ massas
  iguais dispostas regularmente ao longo do seu comprimento (ver a figura). 
\end{minipage}\hfill
\begin{tikzpicture}[baseline=(current bounding box.north)]
  \small
  % Supporting structure
  \fill [pattern = north west lines] (-0.75,0) rectangle ++(1.5,.2);
  \draw[thick] (-0.75,0) -- ++(1.5,0);
  \draw (0.80,0) -- (0.90,0);
  \draw [->] (0.85,0) -- +(0,-1) node[below]{$x$};
  % Spring
  \tikzset
  {
    spring/.style n args={2}{decorate,
                   decoration={coil, aspect=0.4,
                               segment length=#1,
                               amplitude = 2mm,
                               %pre length=2mm,
                               %post length=#2
                              }
                            }
  }
  \draw (0,0) -- (0,-0.4) coordinate(a);
  \draw [spring={1.5mm}{0.5mm}] ([yshift=-1.8mm]a) -- ++(0,-1.45)
    coordinate[yshift=-1.65mm](b);
  \draw [spring={1.0mm}{0mm}] ([yshift=-1.8mm]b)-- ++(0,-0.61) coordinate (c);
  \path (a) node[circle, draw, fill=white,inner sep=1]{1};
  \path (b) node[circle, draw, fill=white,inner sep=1]{2};
  %\fill (c) circle(0.1);
  \path (c) node[inner sep=0,below]{$\vdots$};
  \draw [spring={0.7mm}{1mm}] (0,-3.75) -- ++(0,-0.405)
    node[below,circle,fill=white, draw, inner sep=0]{$N$};
 
\end{tikzpicture}
\end{document}
