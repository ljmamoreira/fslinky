\documentclass{report}
\usepackage[utf8]{inputenc}
\usepackage[portuguese]{babel}
\usepackage[a4paper]{geometry}
\usepackage{icomma, amsmath}
\setlength{\parindent}{0mm}
\begin{document}
\textbf{Título: }Descrição da queda de uma mola real com o modelo de mola ideal

\vspace{.5em}
\textbf{Autores: }João Santos (1), João Esteves (1), Luís JM Amoreira(1,2,3)

\vspace{.5em}
\textbf{Filiação dos autores: }\\
(1) Departamento de Física, UBI\\
(2) LIP - Laboratório de Interações e Partículas (UC)\\
(3) CMA - Centro de Matemática e Aplicações (UBI)

\vspace{.5em}
\textbf{Email de contacto: }\texttt{amoreira@ubi.pt}

\vspace{.5em}
\textbf{Resumo}\\
  Vários vídeos disponíveis na plataforma YouTube mostram a queda de uma mola
  elástica a partir de uma situação de repouso estático em que ela se encontra
  na vertical, suspensa da sua extremidade superior [refs 1]. Estes vídeos são
  interessantes poque mostram a extremidade inferior da mola como que a aguardar
  que a extremidade superior a atinja, antes de começar o seu movimento de queda
  propriamente dito. 
	
  A explicação deste comportamento é dada pela teoria da elasticidade.  A onda
  de deformação gerada na extremidade superior da mola quando é solta propaga-se
  longitudinalmente com uma velocidade finita, e só quando atinge a extremidade
  inferior, alterando aí o estado de deformação inicial, se modifica o
  equilíbrio de forças (peso e força elástica) que mantinham esta extremidade em
  repouso.
	
  Claramente, o modelo elementar de mola ideal, em que se despreza a sua massa,
  é insuficiente para enquadrar esta explicação, uma vez que não tendo massa,
  (1)~a sua deformação é sempre uniforme; logo, a força sobre a extremidade
  inferior altera-se instantaneamente assim que a extremidade superior inicia a
  sua queda, e (2)~a mola não fica sujeita à gravidade, ou seja, nem sequer cai!
  Mas será possível dar conta deste comportamento das molas reais considerando
  molas ideais com massas distribuídas regularmemente ao longo do seu
  comprimento?

  Neste trabalho analiza-se o numericamente (usando a linguagem python e as
  bibliotecas numpy e scipy [refs 2]) o movimento de queda de um sistema formado por $N$
  massas iguais ligadas sequencialmente por molas iguais.

  Esta conclusão foi verificada experimentalmente usando uma mola real de aço e
  bolas de ténis como massas, para os casos $N=2$ e $N=3$. A queda deste sistema
  foi registrada em vído a 120\,fps e analisada com o programa Tracker [ref 3]
  para recolher as posições das diferentes massas como funções do tempo.

  Verifica-se que a queda das molas reais pode de facto ser aproximada
  com este modelo e que, como seria de esperar, a aproximação é tanto melhor
  quanto maior $N$ (mantendo constante a massa total e as características de
  elasticidade da mola). 

  Consideramos que este trabalho permite identificar claramente os elementos
  essenciais do modelo de mola ideal e pôr evidência situações em que ele é
  inapropriado. Simultaneamente, pode ser usado para ilustrar a possibilidade de
  desenvolvimento de teorias de campo como limite de teorias discretas quando o
  número de elementos do sistema em análise tende para um infinito não
  numerável.

\vspace{.5em}
\textbf{Referências}

[ref 1] Youtube

[ref 2] python, numpy, scipy

[ref 3] Tracker



\end{document}
